\documentclass{book}
\title{\textbf{Linear Algebra}}
\author{Meow}
\date{2024-02-24}

\usepackage{amsfonts}
\usepackage{amsmath}
\usepackage{graphicx}
\usepackage{mathabx}
\usepackage{bm} % For bold vectors

\addtolength{\topmargin}{-3cm}
\addtolength{\textheight}{3cm}
\begin{document}
\maketitle
\thispagestyle{empty}

\chapter{Introduction}
\section{Notation}
A vector is an ordered list of numbers:
\[
    \begin{pmatrix}
        -1.1 \\ 0.0 \\ 3.7
    \end{pmatrix}
\]

Vectors are denoted here using boldface: $\bm{a}$, $\bm{b}$, etc. The $i$-th element of an $n$-vector is $a_i$, where $i$ is the index.

Two vectors $\bm{a}, \bm{b}$ are equal if $a_i = b_i$ for all $i$. A \textbf{stacked vector} with vectors $\bm{b}$, $\bm{c}$, $\bm{d}$ is:
\[
    \bm{a} = \begin{pmatrix}
        \bm{b} \\ \bm{c} \\ \bm{d}
    \end{pmatrix}
\]

A \textbf{unit vector} $\bm{e}_i$ has 1 in the $i$-th position and 0 elsewhere:
\[
    \bm{e}_1 = \begin{pmatrix}
        1 \\ 0 \\ 0
    \end{pmatrix}, \quad
    \bm{e}_2 = \begin{pmatrix}
        0 \\ 1 \\ 0
    \end{pmatrix}
\]

A vector is \textbf{sparse} if most entries are 0.

\section{Addition and Scalar Multiplication}
Vector addition is entry-wise:
\[
    \begin{pmatrix} 1 \\ 3 \\ 2 \end{pmatrix} +
    \begin{pmatrix} 2 \\ 4 \\ 6 \end{pmatrix} =
    \begin{pmatrix} 3 \\ 7 \\ 8 \end{pmatrix}
\]

\subsection{Properties of Vector Addition}
\begin{itemize}
    \item Commutative: $\bm{a} + \bm{b} = \bm{b} + \bm{a}$
    \item Associative: $(\bm{a} + \bm{b}) + \bm{c} = \bm{a} + (\bm{b} + \bm{c})$
    \item Identity: $\bm{a} + \bm{0} = \bm{a}$
    \item Inverse: $\bm{a} - \bm{a} = \bm{0}$
\end{itemize}

\subsection{Scalar-Vector Multiplication}
For scalar $\alpha$ and vector $\bm{a}$:
\[
    (-2) \begin{pmatrix} 2 \\ 4 \\ 7 \end{pmatrix} = 
    \begin{pmatrix} -4 \\ -8 \\ -14 \end{pmatrix}
\]

\begin{itemize}
    \item Associative: $\alpha(\beta\bm{a}) = (\alpha\beta)\bm{a}$
    \item Distributive: $(\alpha + \beta)\bm{a} = \alpha\bm{a} + \beta\bm{a}$
    \item Distributive: $\alpha(\bm{a} + \bm{b}) = \alpha\bm{a} + \alpha\bm{b}$
\end{itemize}

\subsection{Linear Combinations \& Inner Product}
A \textbf{linear combination} of vectors $\bm{a}_1, ..., \bm{a}_m$ with scalars $\beta_1, ..., \beta_m$ is:
\[
    \beta_1\bm{a}_1 + \cdots + \beta_m\bm{a}_m
\]

The \textbf{dot product} (inner product) of vectors $\bm{a}, \bm{b}$ is:
\[
    \bm{a} \cdot \bm{b} = a_1b_1 + a_2b_2 + \cdots + a_nb_n
\]

Example:
\[
    \begin{pmatrix} 1 \\ -1 \\ 0 \end{pmatrix} \cdot
    \begin{pmatrix} 2 \\ 9 \\ -4 \end{pmatrix} = 
    (1)(2) + (-1)(9) + (0)(-4) = -7
\]

\subsubsection{Inner Product Properties}
For vectors $\bm{a}, \bm{b}, \bm{c}$ and scalar $\gamma$:
\begin{itemize}
    \item Commutative: $\bm{a} \cdot \bm{b} = \bm{b} \cdot \bm{a}$
    \item Scalar association: $(\gamma\bm{a}) \cdot \bm{b} = \gamma(\bm{a} \cdot \bm{b})$
    \item Distributive: $\bm{a} \cdot (\bm{b} + \bm{c}) = \bm{a} \cdot \bm{b} + \bm{a} \cdot \bm{c}$
    \item Relation to norm: $\bm{a} \cdot \bm{a} = \|\bm{a}\|^2$
\end{itemize}

\end{document}